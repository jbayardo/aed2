\documentclass[a4paper,10pt]{article}
\usepackage[paper=a4paper, hmargin=1.5cm, bottom=1.5cm, top=3.5cm]{geometry}
\usepackage[latin1]{inputenc}
\usepackage[T1]{fontenc}
\usepackage[spanish]{babel}
\usepackage{xspace}
\usepackage{xargs}
\usepackage{ifthen}
\usepackage{aed2-tad,aed2-symb,aed2-itef}


\newcommand{\moduloNombre}[1]{\textbf{#1}}

\let\NombreFuncion=\textsc
\let\TipoVariable=\texttt
\let\ModificadorArgumento=\textbf
\newcommand{\res}{$res$\xspace}
\newcommand{\tab}{\hspace*{7mm}}

\newcommandx{\TipoFuncion}[3]{%
  \NombreFuncion{#1}(#2) \ifx#3\empty\else $\to$ \res\,: \TipoVariable{#3}\fi%
}
\newcommand{\In}[2]{\ModificadorArgumento{in} \ensuremath{#1}\,: \TipoVariable{#2}\xspace}
\newcommand{\Out}[2]{\ModificadorArgumento{out} \ensuremath{#1}\,: \TipoVariable{#2}\xspace}
\newcommand{\Inout}[2]{\ModificadorArgumento{in/out} \ensuremath{#1}\,: \TipoVariable{#2}\xspace}
\newcommand{\Aplicar}[2]{\NombreFuncion{#1}(#2)}

\newlength{\IntFuncionLengthA}
\newlength{\IntFuncionLengthB}
\newlength{\IntFuncionLengthC}
%InterfazFuncion(nombre, argumentos, valor retorno, precondicion, postcondicion, complejidad, descripcion, aliasing)
\newcommandx{\InterfazFuncion}[9][4=true,6,7,8,9]{%
  \hangindent=\parindent
  \TipoFuncion{#1}{#2}{#3}\\%
  \textbf{Pre} $\equiv$ \{#4\}\\%
  \textbf{Post} $\equiv$ \{#5\}%
  \ifx#6\empty\else\\\textbf{Complejidad:} #6\fi%
  \ifx#7\empty\else\\\textbf{Descripción:} #7\fi%
  \ifx#8\empty\else\\\textbf{Aliasing:} #8\fi%
  \ifx#9\empty\else\\\textbf{Requiere:} #9\fi%
}

\newenvironment{Interfaz}{%
  \parskip=2ex%
  \noindent\textbf{\Large Interfaz}%
  \par%
}{}

\newenvironment{Representacion}{%
  \vspace*{2ex}%
  \noindent\textbf{\Large Representación}%
  \vspace*{2ex}%
}{}

\newenvironment{Algoritmos}{%
  \vspace*{2ex}%
  \noindent\textbf{\Large Algorítmos}%
  \vspace*{2ex}%
}{}


\newcommand{\Titulo}[1]{
  \vspace*{1ex}\par\noindent\textbf{\large #1}\par
}

\newenvironmentx{Estructura}[2][2={estr}]{%
  \par\vspace*{2ex}%
  \TipoVariable{#1} \textbf{se representa con} \TipoVariable{#2}%
  \par\vspace*{1ex}%
}{%
  \par\vspace*{2ex}%
}%

\newboolean{EstructuraHayItems}
\newlength{\lenTupla}
\newenvironmentx{Tupla}[1][1={estr}]{%
    \settowidth{\lenTupla}{\hspace*{3mm}donde \TipoVariable{#1} es \TipoVariable{tupla}$($}%
    \addtolength{\lenTupla}{\parindent}%
    \hspace*{3mm}donde \TipoVariable{#1} es \TipoVariable{tupla}$($%
    \begin{minipage}[t]{\linewidth-\lenTupla}%
    \setboolean{EstructuraHayItems}{false}%
}{%
    $)$%
    \end{minipage}
}

\newcommandx{\tupItem}[3][1={\ }]{%
    %\hspace*{3mm}%
    \ifthenelse{\boolean{EstructuraHayItems}}{%
        ,#1%
    }{}%
    \emph{#2}: \TipoVariable{#3}%
    \setboolean{EstructuraHayItems}{true}%
}

\newcommandx{\RepFc}[3][1={estr},2={e}]{%
  \tadOperacion{Rep}{#1}{bool}{}%
  \tadAxioma{Rep($#2$)}{#3}%
}%

\newcommandx{\Rep}[3][1={estr},2={e}]{%
  \tadOperacion{Rep}{#1}{bool}{}%
  \tadAxioma{Rep($#2$)}{true \ssi #3}%
}%

\newcommandx{\Abs}[5][1={estr},3={e}]{%
  \tadOperacion{Abs}{#1/#3}{#2}{Rep($#3$)}%
  \settominwidth{\hangindent}{Abs($#3$) \igobs #4: #2 $\mid$ }%
  \addtolength{\hangindent}{\parindent}%
  Abs($#3$) \igobs #4: #2 $\mid$ #5%
}%

\newcommandx{\AbsFc}[4][1={estr},3={e}]{%
  \tadOperacion{Abs}{#1/#3}{#2}{Rep($#3$)}%
  \tadAxioma{Abs($#3$)}{#4}%
}%


\newcommand{\DRef}{\ensuremath{\rightarrow}}

\begin{document}

%%%%%%%%%%%%%%%%%%%%%%%%%%%%%%%%%%%%%%%%%%%%%%%%%%%%%%%%%%%%%%%%%%%%%%%%%%%%%%%
%%%%						Aca empieza el documento
%%%%%%%%%%%%%%%%%%%%%%%%%%%%%%%%%%%%%%%%%%%%%%%%%%%%%%%%%%%%%%%%%%%%%%%%%%%%%%%

\section{Aclaraciones}


%%%%%%%%%%%%%%%%%%%%%%%%%%%%%%%%%%%%%%%%%%%%%%%%%%%%%%%%%%%%%%%%%%%%%%%%%%%%%%%
%%%%								Trie
%%%%%%%%%%%%%%%%%%%%%%%%%%%%%%%%%%%%%%%%%%%%%%%%%%%%%%%%%%%%%%%%%%%%%%%%%%%%%%%
\section{Módulo Diccionario Trie($\alpha$)}

\begin{Interfaz}
\end{Interfaz}

\begin{Representacion}
\end{Representacion}

%%%%%%%%%%%%%%%%%%%%%%%%%%%%%%%%%%%%%%%%%%%%%%%%%%%%%%%%%%%%%%%%%%%%%%%%%%%%%%%
%%%%							Cola de Prioridad
%%%%%%%%%%%%%%%%%%%%%%%%%%%%%%%%%%%%%%%%%%%%%%%%%%%%%%%%%%%%%%%%%%%%%%%%%%%%%%%
\section{Módulo Cola de Prioridad Heap($\alpha$)}

\begin{Interfaz}
\end{Interfaz}

\begin{Representacion}
\end{Representacion}

%%%%%%%%%%%%%%%%%%%%%%%%%%%%%%%%%%%%%%%%%%%%%%%%%%%%%%%%%%%%%%%%%%%%%%%%%%%%%%%
%%%%								AVL
%%%%%%%%%%%%%%%%%%%%%%%%%%%%%%%%%%%%%%%%%%%%%%%%%%%%%%%%%%%%%%%%%%%%%%%%%%%%%%%
\section{Módulo Conjunto AVL($\alpha$)}

\begin{Interfaz}

  \textbf{parámetros formales}\hangindent=2\parindent\\
  \parbox{1.7cm}{\textbf{géneros}} $\alpha$\\
  \parbox[t]{1.7cm}{\textbf{función}}\parbox[t]{\textwidth-2\parindent-1.7cm}{%
    \InterfazFuncion{Copiar}{\In{a}{$\alpha$}}{$\alpha$}
    {$res \igobs a$}
    [$\Theta(copy(a))$]
    [función de copia de $\alpha$'s]
  }

  \textbf{se explica con}: \tadNombre{Secuencia$(\alpha)$}, \tadNombre{Iterador Bidireccional($\alpha$)}.

  \textbf{géneros}: \TipoVariable{lista$(\alpha)$}, \TipoVariable{itLista($\alpha$)}.

  \Titulo{Operaciones básicas de lista}

  \InterfazFuncion{Vacía}{}{lista$(\alpha)$}%
  {$res \igobs \secuencia{}$}%
  [$\Theta(1)$]
  [genera una lista vacía.]

  \InterfazFuncion{AgregarAdelante}{\Inout{l}{lista($\alpha$)}, \In{a}{$\alpha$}}{itLista($\alpha$}
  [$l \igobs l_0$]
  {$l \igobs \secuencia{a}[l_0]$ $\land$ $res$ $=$ CrearItBi(\secuencia{}, $l$) $\land$ alias(SecuSuby($res$) $=$ $l$)}
  [$\Theta(copy(a))$]
  [agrega el elemento $a$ como primer elemento de la lista.  Retorna un iterador a $l$, de forma tal que Siguiente devuelva $a$.]
  [el elemento $a$ agrega por copia. El iterador se invalida si y sólo si se elimina el elemento siguiente del iterador sin utilizar la función \NombreFuncion{EliminarSiguiente}.]

 
  \Titulo{Operaciones del iterador}

  \InterfazFuncion{CrearIt}{\In{l}{lista($\alpha$)}}{itLista($\alpha$)}
  {$res$ $\igobs$ crearItBi(\secuencia{}, $l$) $\land$ alias(SecuSuby($it$) $=$ $l$)}
  [$\Theta(1)$]
  [crea un iterador bidireccional de la lista, de forma tal que al pedir \NombreFuncion{Siguiente} se obtenga el primer elemento de $l$.]
  [el iterador se invalida si y sólo si se elimina el elemento siguiente del iterador sin utilizar la función \NombreFuncion{EliminarSiguiente}.]

  \InterfazFuncion{CrearItUlt}{\In{l}{lista($\alpha$)}}{itLista($\alpha$)}
  {$res$ $\igobs$ crearItBi($l$, \secuencia{}) $\land$ alias(SecuSuby($it$) $=$ $l$)}
  [$\Theta(1)$]
  [crea un iterador bidireccional de la lista, de forma tal que al pedir \NombreFuncion{Anterior} se obtenga el último elemento de $l$.]  
  [el iterador se invalida si y sólo si se elimina el elemento siguiente del iterador sin utilizar la función \NombreFuncion{EliminarSiguiente}.]

\end{Interfaz}

\begin{Representacion}
\end{Representacion}

%%%%%%%%%%%%%%%%%%%%%%%%%%%%%%%%%%%%%%%%%%%%%%%%%%%%%%%%%%%%%%%%%%%%%%%%%%%%%%%
%%%%							Restricción
%%%%%%%%%%%%%%%%%%%%%%%%%%%%%%%%%%%%%%%%%%%%%%%%%%%%%%%%%%%%%%%%%%%%%%%%%%%%%%%
\section{Módulo Restricción}

\begin{Interfaz}
\end{Interfaz}

\begin{Representacion}
\end{Representacion}

%%%%%%%%%%%%%%%%%%%%%%%%%%%%%%%%%%%%%%%%%%%%%%%%%%%%%%%%%%%%%%%%%%%%%%%%%%%%%%%
%%%%								Grafo
%%%%%%%%%%%%%%%%%%%%%%%%%%%%%%%%%%%%%%%%%%%%%%%%%%%%%%%%%%%%%%%%%%%%%%%%%%%%%%%
\section{Módulo Grafo}

\begin{Interfaz}
\end{Interfaz}

\begin{Representacion}
\end{Representacion}

%%%%%%%%%%%%%%%%%%%%%%%%%%%%%%%%%%%%%%%%%%%%%%%%%%%%%%%%%%%%%%%%%%%%%%%%%%%%%%%
%%%%								Ciudad
%%%%%%%%%%%%%%%%%%%%%%%%%%%%%%%%%%%%%%%%%%%%%%%%%%%%%%%%%%%%%%%%%%%%%%%%%%%%%%%
\section{Módulo Ciudad}

\begin{Interfaz}
\end{Interfaz}

\begin{Representacion}
\end{Representacion}

\end{document}
