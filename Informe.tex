\documentclass[a4paper,titlepage]{report}

\usepackage{fullpage}

\usepackage[T1]{fontenc}
\usepackage[utf8]{inputenc}
\usepackage[spanish]{babel}

\usepackage{listings}
\usepackage{color}

\definecolor{dkgreen}{rgb}{0,0.6,0}
\definecolor{gray}{rgb}{0.5,0.5,0.5}
\definecolor{mauve}{rgb}{0.58,0,0.82}

\lstset{frame=tb,
  language=Java,
  aboveskip=3mm,
  belowskip=3mm,
  showstringspaces=false,
  columns=flexible,
  basicstyle={\small\ttfamily},
  numbers=none,
  numberstyle=\tiny\color{gray},
  keywordstyle=\color{blue},
  commentstyle=\color{dkgreen},
  stringstyle=\color{mauve},
  breaklines=true,
  breakatwhitespace=false
  tabsize=4
}

\begin{document}

\title{Algorítmos y Estructuras de Datos II\\
Trabajo Práctico 1\\
Grupo 6}

\author{
	Bayardo, Julián\\
	\texttt{850/13}
	\and
	Cuneo, Christian\\
	\texttt{755/13}
	\and
	Gambaccini, Ezequiel\\
	\texttt{715/13}
	\and
	Lebrero Rial, Ignacio Manuel\\
	\texttt{751/13}
}

\date{9 de Septiembre del 2014}

\maketitle

\section{Aclaraciones}

Para comprender perfectamente nuestro modelado hay que tener en cuenta estos puntos:

La ciudad está formada por un mapa, los robots que están en circulación en el sistema y los movimientos de estos robots desde que ingresaron al sistema. El mapa no puede cambiar una vez que el sistema comenzó.

El mapa que modelamos comprende que una estación se puede conectar a otra estación, o a si misma, por una cantidad no limitada de sendas, todas estas deben tener restricciones distintas. También puede haber estaciones sin sendas conectadas, y estas no serán estaciones bloqueantes bajo ninguna circunstancia.

Las restricciones de cada senda están modeladas en forma de árbol donde todas las hojas son características y los nodos son conectores lógicos.

El sistema asigna RURs desde el 1 de forma incremental a los robots que ingresan al sistema, nunca se van a reutilizar RURs de robots quitados de circulación.

Un robot elige la senda a circular (siempre y cuando sea un movimiento valido), pudiendo inclusive ir por una senda que cuente una infracción aun si hay sendas disponibles que no le cuenten infracciones.
Como aclara el enunciado, los robots se identifican sólo por el RUR, por lo tanto la igualdad entre dos robots es juzgada sólo por ese observador. 

Se toma que dos ciudades son iguales cuando el mapa es igual observacionalmente y cuando el conjunto de robots en circulación (y sus historiales) son los mismos observacionalmente, no se tienen en cuenta los robots que hayan sido quitados de circulación a lo largo del funcionamiento del sistema, ni sus historiales. 

En este sistema no se modelo el tiempo, esta permitido en cualquier momento llevar a cabo una inspección.

\section{TAD}
\lstinputlisting{tp.tad}

\end{document}
