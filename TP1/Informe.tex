\documentclass[a4paper,titlepage]{article}

\usepackage{fullpage}

\usepackage[T1]{fontenc}
\usepackage[utf8]{inputenc}
\usepackage[spanish]{babel}
\usepackage{pdfpages}

\begin{document}

\title{Algorítmos y Estructuras de Datos II\\
Trabajo Práctico 1 (Especificación)\\
Grupo 6}

\author{
	Bayardo, Julián\\
	julian@bayardo.com.ar\\
	\texttt{850/13}
	\and
	Cuneo, Christian\\
	chriscuneo93@gmail.com\\
	\texttt{755/13}
	\and
	Gambaccini, Ezequiel\\
	ezequiel.gambaccini@gmail.com\\
	\texttt{715/13}
	\and
	Lebrero Rial, Ignacio Manuel\\
	ignaciolebrero@gmail.com\\
	\texttt{751/13}
}

\date{9 de Septiembre del 2014}

\maketitle

\section{Aclaraciones}

Asumimos que las estaciones, junto con las sendas y las restricciones entre ellas no son mutables: una instancia del TAD Ciudad comienza con un mapa definido y no puede ser cambiado. Suponemos también que las estaciones pueden tener sendas que las conecten a sí mismas, mas sólo una senda que la conecte a otra estación definida. Aparte, consideramos que pueden haber estaciones no conexas, y en cuyo caso no son consideradas bloqueantes (aunque los robots en ellas no podrían moverse hacia ningún lado).

Los RUR se suponen asignados al momento de agregar un robot, la unica restricción al respecto es que el RUR efectivamente sea único. No puede existir un robot que no tenga ninguna característica.

Dos restricciones son equivalentes cuando su tabla de verdad es la misma para todo posible conjunto de características.

Cabe destacar que no modelamos el paso del tiempo, por lo que las inspecciones no son automáticas, sino que deben ser causadas deliberadamente por quien maneje el sistema.

\section{Sobre el formato}

Inicialmente escribimos la especificación utilizando UTF-8 en lugar de LaTeX, pensando que podríamos fácilmente incluir el archivo en nuestro informe, mas esto resultó ser problemático por varias cuestiones. Finalmente, optamos por incluirlo por separado como un archivo en otro pdf. Debido a problemas con esta solución, tuvimos que ajustar el texto para que no pase de 90 columnas; adoptamos como convención que si las precondiciones de una función superan tal tamaño, la guarda se pasa a otra linea y se indenta.

\includepdf[pages={1, 2, 3, 4}]{Especificacion.pdf}

\end{document}
