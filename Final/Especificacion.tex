%% Compilar usando PDFLaTeX
\documentclass[10pt,a4paper,notitlepage]{article}
\usepackage[spanish]{babel}
\usepackage[utf8]{inputenc}
\usepackage[T1]{fontenc}

\usepackage{amsmath}
\usepackage{amsthm}
\usepackage{amsfonts}
\usepackage{amssymb}
\usepackage{algorithm}
\usepackage{algpseudocode}
\usepackage{graphicx}

\newtheorem{deftn}{Definición}
\newtheorem{defthm}{Teorema}

\author{Julián Bayardo}
\title{Resumen de Especificación}
\begin{document}



Daban un enunciado y un TAD ya especificado y había que encontrar errores. Había funciones que violaban la congruencia, dado el problema a modelar faltaban otros observadores y generadores, también faltaban restricciones en algunas funciones, había un TAD auxiliar que era dependiente del TAD que se estaba modelando, etc.

Explicar las consecuencias en el desarrollo de un software si se tienen en la especificacion los siguientes errores:
Funciones inconsistentes
Funciones que violan la congruencia
Funciones que cumplen con todo pero tienen nombres de una letra
Etc.

Daban 5 errores comunes en los TADS y habia que explicar cual era su consecuencia en el diseño. No recuerdo todos, pero algunos eran:
Incongruencia
Inconsistencia (contradiccion logica)
Axiomatizacion muy complicada
Sub Especificacion
Se usan muchos cuantificadores en la axiomatizacion
Tambien habia que decir cual era el error menos grave.

Responda justificando.
¿Cuándo decimos que una función no es congruente con la igualdad observacional?
Cuando diferencia más instancias que los observadores básicos.
Cuando es redundante con respecto a los observadores básicos.
Ambas.
Ninguna.
¿Cuál es la diferencia entre tipo y género (por ejemplo, entre NAT y nat)?
¿Qué criterio utilizaría para definir si una operación puede ser generador o ser otra operación?

Dada la siguiente especificación de un TAD..
a. Explicar en palabras las características principales del tipo.
b. Escribir la igualdad observacional.
c. El tipo tal como está especificado tiene un problema (podríamos decir que está incorrectamente especificado)) Indique cuál es y proponga una solución.

Dada la siguiente especificación sobre una cena con participantes, encuentre los errores y corríjalos (axiomatice o describa como arreglar el error)
Tad Persona
generadores
nueva edad × dni → persona
observadores
. = . persona × persona → bool

Tad Cena
generadores
crear conj(personas) → cena
llega_invitado persona × plato x × cena → cena
observadores
invitados cena → conj(personas)
que_plato_trajo? persona p × cena c → plato { p ∊ invitados(c) }
suma_de_edades cena → nat
Tad Regalo es Nat

Responda verdadero o falso justificando
Lo que hace que una operación sea observador básico es que deba escribirse en base a los generadores.
Si una operación rompe la congruencia debe ser transformada en observador básico
Dos instancias del mismo TAD pueden ser observacionalmente iguales y aún así ser distingibles por una operación.
Si un enunciado dice "siempre que sucede A sucede inmeditamente B y B no puede suceder de ninguna otra manera" y la correspondiente axiomatización incluye las operaciones A y B entonces el TAD está mal escrito

Explicar qué es un observador básico y cuál es su utilidad en la especificación de TADs.

El sistema de especificación TADSOB es como el de los TAD pero no tiene observadores básicos. Cuenta solamente con generadores y operaciones en general. Piense en los TADs básicos que conoce y discuta las limitaciones de los TADSOB.

Se tiene el siguiente TAD:
TAD PUNTO
generadores:
comenzar: → punto
subir: punto × nat → punto
derecha: punto × nat → punto

observadores:
X: punto → nat
Y: punto → nat

otras operaciones:
mover: punto × nat n × nat m → punto

axiomas:
X(comenzar) = 0
Y(comenzar) = 0
X(subir(p,n)) = X(p)
Y(subir(p,n)) = Y(p)+n
X(derecha(p,n)) = X(p)+n
Y(derecha(p,n)) = Y(p)
mover(p,n,m) = subir(derecha(p,n),m)

¿Se puede plantear la demostración por inducción estructural de una propiedad sobre el TAD punto utilizando en los teoremas a demostrar sólo comenzar(), mover(), X() e Y()? Justifique.

\end{document}